\documentclass[12pt, a4paper, oneside]{ctexart}
\usepackage{amsmath, amsthm, amssymb, appendix, bm, graphicx, hyperref, mathrsfs}

\title{\textbf{论文标题}}
\author{Dylaaan}
\date{\today}
\linespread{1.5}
\newtheorem{theorem}{定理}[section]
\newtheorem{definition}[theorem]{定义}
\newtheorem{lemma}[theorem]{引理}
\newtheorem{corollary}[theorem]{推论}
\newtheorem{example}[theorem]{例}
\newtheorem{proposition}[theorem]{命题}
\renewcommand{\abstractname}{\Large\textbf{摘要}}
\usepackage{float} % 启用 [H] 位置选项

\begin{document}
	
	\maketitle
	
	\setcounter{page}{0}
	\maketitle
	\thispagestyle{empty}
	
	\begin{abstract}
		这里是摘要. 
		\par\textbf{关键词:}这里是关键词; 这里是关键词. 
	\end{abstract}
	
	\newpage
	\pagenumbering{Roman}
	\setcounter{page}{1}
	\tableofcontents
	\newpage
	\setcounter{page}{1}
	\pagenumbering{arabic}
	
	\section{一级标题}
	
	
	
	

\begin{figure}[H]
	\centering
	\includegraphics[width=0.7\linewidth]{screenshot001}
	\caption{}
	\label{fig:screenshot001}
\end{figure}

推导过程:
\begin{align*}
	\Delta L = n_2 \cdot(AO + AB) -n_1\cdot OC\\
	AB = AO = \frac{d}{n_2 \cdot\sin\theta } \\
	OC = OB \cdot \tan\alpha \\
	n_1 \sin \alpha = n_2 \sin \theta \\
	n_1 = 1
\end{align*}
求解得到:\\
$$\Delta L = 2\cdot d \cdot \sqrt{n_2^2 - \sin \alpha^2} $$  \\
相干光发生条件:
\begin{align*}
	\frac{2\pi}{\lambda}\Delta L = 2\cdot m\cdot \pi  \\
	\frac{2\pi}{\lambda}\Delta L = (2\cdot m + 1 )\cdot \pi
\end{align*}
结论(带初相位):
$$ d = \frac{m (\lambda + \varphi)  }{2\sqrt{n_2^2 - \sin \alpha^2}}  $$
在实际中,我们通过观测数据的两个峰值的间距,或者波谷(把波长换算为波数)。有:
$$ 2v_1d\sqrt{n_2^2 - \sin \alpha^2} = m  $$
$$ 2v_2d\sqrt{n_2^2 - \sin \alpha^2} = m+1  $$
两个式子做差:\\
令: $$|v_1 - v_2| = \Delta T$$
得到:$$d = \frac{1}{2 \Delta T \sqrt{n_2^2 - \sin \alpha^2}}$$
综上所述:我们只需要得到相邻波峰的横轴的差,折射率$n_2$,以及入射角,就可以得到厚度d
	 
	\section{第二问}
	\subsection{模型建立}
	\begin{align*}
		\frac{2\pi}{\lambda}\Delta L = 2\cdot m\cdot \pi  \\
	\end{align*}
	阐释1:为什么相邻的两个波峰的波数可以用来求d
	解释一下模型。
	\subsubsection{证明为相干光}
	同一光源发出,频率相同;反射光和折射光平行。
	\subsubsection{具有波长差,发生干涉}
	对于附件1中图像的周期性变化,是由于反射光与经过折射的反射光具有一定的光路差,相干涉,从而导致了图像的变化,
	\subsection{数据预处理}
	\subsubsection{除去干扰}
	通过matplot库函数,画出附件一和附件二的图像。我们明显看到,附件一在波数在500-1000有着明显的突变。
	通过查找资料,(此处省略一万字),我们可以直接省略掉波数小于1500的部分,同时也不会失去模型的准确度。
	\subsubsection{提取周期}
	傅里叶变换可以将时域信号转换为频域信号,通过分析频域特征,能够有效推测出数据中的周期成分 。利用傅里叶变换就可以清晰地找到这些周期信号,进而分离出整体趋势和周期扰动。
	\subsubsection{计算折射率}
	题目明确提到外延层的折射率通常不是常数,它与掺杂载流子的浓度、红外光谱的波长等参数有关。由于附件1和附件2使用的是同一块块碳化硅晶圆片,故可以假设载流子浓度一定,折射率的主要影响因素为红外光谱的波长。菲涅耳公式描述了光在不同介质分界面上的反射和折射的振幅、相位等关系。能够结合光的反射、折射等光学现象的观测数据来得到折射率。\\
	我们通过一定的处理得到(省略两万字)折射率函数
	\subsubsection{计算d}
	我们提取出了附件1函数的周期性变化的周期,通过(什么方法)得到 最优 T
	通过最优T
	\newpage
	
	\begin{thebibliography}{99}
		\bibitem{a}作者. \emph{文献}[M]. 地点:出版社,年份.
		\bibitem{b}作者. \emph{文献}[M]. 地点:出版社,年份.
	\end{thebibliography}
	
	\newpage
	
	\begin{appendices}
		\renewcommand{\thesection}{\Alph{section}}
		\section{附录标题}
		这里是附录. 
	\end{appendices}
	
\end{document}
